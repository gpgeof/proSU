%%%%%%%%%%%%%%%%%%%%%%%%%%%%%%%%%%%%%%%%
% \begin{agradecimentos}
% 
% \vspace{+3.0cm}

% Ao orientador, Prof. Dr. Lourenildo W. B. Leite, por todo o conhecimento e experi\^{e}ncia repassada ao longo destes anos de conviv\^{e}ncia, e por todo o apoio para que este trabalho se tornasse realizável.
% 
% Ao Dr. Wildney W. S. Vieira por sua colabora\c{c}\~{a}o neste trabalho com uma ajuda preciosa na programação computacional e no processamento de dados sísmicos.
% 
% % Ao Prof. Dr. Daniel Leal Macedo e ao Prof. Dr. José Jadsom Sampaio de Figueiredo por participarem da Banca Examinadora, e pelas sugestões para a melhora deste relatório e sua apresentação.
% 
% Ao corpo docente da Graduação em Geofísica da UFPA por todo o conhecimento repassado para o meu desenvolvimento intelectual, científico e técnico.
% 
% Aos meus pais Modesto Diociesse e Joana Cunha pelo apoio familiar nestes anos de estudo.
% 
% Aos familiares que contribuíram de maneira direta para a execução deste trabalho, em especial meu tio Doval Cunha por todo apoio durante minha graduação, a meus avós Manoel Souza e Teresinha Cunha pelo carinho e cuidado.
% 
% Aos amigos Larissa Almeida, Alexandre Cavalcante, Arthur Rodrigues, Sheyla Machado, Melina Silva pelo apoio neste trabalho.
% 
% Aos amigos de graduação e da pós graduação em geofísica, e em especial Gabriel Bricio, Antônio Leite, Rayle Cristine, Karolina Correia, Cesar Pessoa, Fernando Andrade e Igor de Jesus.
% 
% Agradeço ao Bom Deus.

% \end{agradecimentos}

%%%%%%%% EPIGRAFE - OPCIONAL %%%%%%%%%%%%%%%%%%%%%%%%%%%%%%%%
%\newpage
%\thispagestyle{empty}
%\vspace*{\fill}
%\begin{citacao}
%	Jesus é a salvação! Jesus é a salvação! Jesus é a salvação! Jesus é a salvação! Jesus é a salvação! Jesus é a salvação! Jesus é a salvação! Jesus é a salvação! Jesus é a salvação! Jesus é a salvação! Jesus é a salvação! Jesus é a salvação! Jesus é a salvação! Jesus é a salvação! Jesus é a salvação! Jesus é a salvação! Jesus é a salvação! Jesus é a salvação! Jesus é a salvação! Jesus é a salvação! Jesus é a salvação! Jesus é a salvação! Jesus é a salvação! Jesus é a salvação! Jesus é a salvação! Jesus é a salvação! Jesus é a salvação! Jesus é a salvação! Jesus é a salvação! Jesus é a salvação! Jesus é a salvação! Jesus é a salvação! Jesus é a salvação! Jesus é a salvação! Jesus é a salvação! Jesus é a salvação! 
%	
%	{\hfill Albert Einstein.}
%\end{citacao}
%
%\newpage
%
%%%%%%%%% RESUMO - OBRIGATORIO %%%%%%%%%%%%%%%%%%%%%%%%%%%%%%%
% \begin{resumo}
% O conteúdo de uma seção sísmica pode ser resumido em ruídos diversos, reflexões, difrações e artefatos. 
% Para esta análise tomamos por base seções sísmicas marinhas da bacia do Jequitinhonha, localizada ao leste do estado da Bahia. 
% O objetivo geral deste trabalho foi o estudo e a aplicação de princípios básicos e fundamentais que descrevem o fenômeno de difração do ponto de vista sísmico para a modelagem e imageamento da subsuperfície, e tomamos como literatura inicial o trabalho clássico do autor Trorey.
% Como objetivo específico foi desenvolver programas em Matlab para a modelagem por difração para se obter as respostas sísmicas de refletores finitos simulando situações típicas encontradas na geologia, como estratos, grabens, horsts, falhas. 
% Neste trabalho nos limitamos a analisar segmentos horizontais curtos e pontuais, e levando em consideração apenas seções afastamento-nulo. 
% Foram desenvolvidos nove modelos geológicos para aplicar a teoria de Trorey, e dentre eles, dois modelos baseados na linha sísmica $214-266$ da bacia do Jequitinhonha. 
% A teoria de Trorey para o fenômeno difração-reflexão é consistente, elegante e simples para explicar feições e artefatos das seções sísmicas, e se baseia no teorema de Green no espaço que explica devidamente o fenômeno físico.
% Para sistematizar o estudo, foram também aplicadas as migrações no tempo dos tipos Kirchhoff e por deslocamento-de-fase (ambas usando o SU), com a finalidade de obter imagens consistentes com a teoria que descreve o colapso de difrações.
% Como conclusão, se observou na implementação que o método é simples para aplicação, e que se pode modelar rapidamente uma seção sísmica com muitos detalhes estruturais, como falhas, sinclinais, anticlinais, horsts, grabens, desde que se discretize bem a estrutura no domínio geológico (espaço). 
% Outra observação muito importante nas conclusões foi sobre a geração de artefatos nas seções modeladas pelo encurvamento da frente de onda ao redor da borda da estrutura, o que caracteriza as difrações. 
% Como trabalho futuro podemos considerar o desenvolvimento da modelagem e inversão de seções sísmicas complexas baseadas em interpretações geológicas importantes na exploração de petróleo, com a finalidade de melhor explicar esta interpretação de forma fisicamente (sismicamente) plausível e consistente, que é um dos objetivos da geofísica.
% \end{resumo}

%%%%%%%%% ABSTRACT - OBRIGATORIO %%%%%%%%%%%%%%%%%%%%%%%%%%%%%%%
% \begin{abstract}
% The contents of a seismic section can be summarized in various noises, reflections, diffractions and artifacts.
% For this analysis we took as a base marine seismic sections of the basin called Jequitinhonha, located to the east of the state of Bahia. The general objective was the study and application of basic and fundamental principles that describe the phenomenon of diffraction from the seismic point of view for the modeling and imaging of the subsurface according to the classic author Trorey.
% As a specific objective was to develop Matlab programs for diffraction modeling to obtain the seismic responses of finite reflectors simulating typical situations found in geology, such as strata, grabens, horsts, faults.
% However, the present work was limited to analyzing short horizontal segments, and taking into account only the null-back sections. Nine geological models were developed to apply Trorey's theory. Among them, two models based on the seismic line $214-266$ of the Jequitinhonha basin. Trorey's theory for the diffraction-reflection phenomenon is consistent, elegant, and simple to explain features and artifacts of seismic sections, and is based on Green's theorem in space that properly explains the physical phenomenon.
% In order to systematize the study, the Kirchhoff-type and phase-shift migrations (both using SU) were also applied, in order to obtain images consistent with the theory that describes the collapse of diffraction.
% As a conclusion, it was observed in the implementation that the method is simple for application, and that a seismic section can be quickly modeled with many structural details, such as faults, synclines, anticlines, horsts, grabens, provided the structure is well discretized in the geological domain (space).
% Another very important observation in the conclusions was about the generation of artifacts in the sections modeled by curving the wave front around the edge of the structure, which characterizes the diffractions.
% As future work we can consider the development of the modeling of complex seismic sections based on important geological interpretations in oil exploration, in order to better explain this interpretation in a physically (seismically) plausible and consistent way, which is one of the objectives of geophysics.
% \end{abstract}

