\appendix
% \addcontentsline{doc}{chapter}{Ap\^endices}
% \section*{Appendices}
\addcontentsline{toc}{section}{Ap\^endices}
\renewcommand{\thesubsection}{\Alph{subsection}}
\label{apendices}

\subsection{SHELL SCRIPT MODELAGEM TRIMODEL}
\label{apendice_A}

Este shell script foi escrito para utilizar a subrotina \textit{trimodel} do pacote Seismic Unix (SU). 
O script gera a figura (2.2) do modelo de camadas curvas.

\lstinputlisting[language=bash,breaklines=true]{codes/geomodel.sh}

\newpage
\subsection{SHELL SCRIPT MODELO DE VELOCIDADE EXATO}
\label{apendice_B}

Este shell script foi escrito para utilizar a subrotina \textit{tri2uni} do pacote Seismic Unix (SU) em conjunto com o Matlab. 
O script gera a figura (2.3) do modelo de camadas curvas.

\lstinputlisting[language=bash,breaklines=true]{codes/modelo_vel_unif.sh}

\subsubsection*{SCRIPT DE CONVERSÃO DO MODELO DE VAGAOROSIDADE}
Script executado para conversão do modelo de vagarosidade em velocidade. Este script de conversão é executado no shell script acima automaticamente.

\lstinputlisting[breaklines=true]{codes/sloth2velocity.m}
\newpage
\subsection{SHELL SCRIPT AQUISIÇÃO TRISEIS}
\label{apendice_C}

Este shell script foi escrito para utilizar a subrotina \textit{triseis} do pacote Seismic Unix (SU). 
O script gera a seção (binário) para o processamento.

\lstinputlisting[language=bash,breaklines=true]{codes/acquisition.sh}
\newpage
\subsection{SHELL SCRIPT GEOMETRIA}
\label{apendice_D}

Este shell script foi escrito para utilizar principalmente a  subrotina \textit{suchw} e \textit{susort} do pacote Seismic Unix (SU). 
O script organiza a geometria da aquisição.

\lstinputlisting[language=bash,breaklines=true]{codes/geometria.sh}
\newpage
\subsection{SHELL SCRIPT PLOT AFASTAMENTO MÍNIMO}
\label{apendice_E}

Este shell script foi escrito para utilizar principalmente a subrotina \textit{supsimage} do pacote Seismic Unix (SU).
O script gera a figura (2.6) do modelo de camadas curvas.

\lstinputlisting[language=bash,breaklines=true]{codes/plot_min_offset.sh}
\newpage
\subsection{SCRIPT MATLAB TRANSFORMADA DE RADON LINEAR}
\label{apendice_F}
Este script em matlab foi escrito para realizar a transformada de Radon linear. O script gera 2 figuras referentes ao CDP-$1000$.

\lstinputlisting[breaklines=true]{codes/main_LRT.m}

\newpage
\subsection{SCRIPT MATLAB FILTRAGEM FK}
\label{apendice_G}
Este script em matlab foi escrito para realizar a filtragem $f-k$. O script gera 12 figuras para o modelo de camadas curvas.
Os parâmetros de entrada são os seguintes:

\lstinputlisting[breaklines=true]{codes/Filtro_corte_FK.m}
\newpage
\subsection{SHELL SCRIPT MÁXIMA COBERTURA}
\label{apendice_H}
Este shell script foi escrito para utilizar principalmente a subrotina \textit{sukeycount} do pacote Seismic Unix (SU).
O script gera a figura (2.23) do modelo de camadas curvas.

\lstinputlisting[language=bash,breaklines=true]{codes/cdp_iluminado.sh}

\subsubsection*{SCRIPT DE PLOTAGEM NO MATLAB}
Após a execução do shell script ``máxima cobertura'' é gerado um arquivo \textit{cdp.txt} que é a entrada da plotagem no matlab. Este arquivo precisa conter apenas os dois vetores colunas que serão carregados no matlab, é necessário excluir o texto gerado no arquivo de saída \textit{cdp.txt}.

\lstinputlisting[breaklines=true]{codes/cdp_iluminado.m}
\newpage
\subsection{SHELL SCRIPT PLOT DOS CDP'S}
\label{apendice_I}
Este shell script foi escrito para utilizar principalmente a subrotina \textit{supswigp} do pacote Seismic Unix (SU).
O script gera as figuras (2.24) e (2.25) do modelo de camadas curvas.

\lstinputlisting[language=bash,breaklines=true]{codes/showcmp.sh}
\newpage
\subsection{SHELL SCRIPT RUÍDO NA SEÇÃO SÍSMICA}
\label{apendice_J}
Este shell script foi escrito para utilizar principalmente a subrotina \textit{suaddnoise} do pacote Seismic Unix (SU).

\lstinputlisting[language=bash,breaklines=true]{codes/ruido.sh}
\newpage
\subsection{SHELL SCRIPT ANÁLISE DE VELOCIDADE ITERATIVA}
\label{apendice_K}
Este shell script foi escrito para utilizar principalmente a subrotina \textit{suvelan} do pacote Seismic Unix (SU).

\lstinputlisting[language=bash,breaklines=true]{codes/iva.scr}
\lstinputlisting[language=bash,breaklines=true]{codes/iva.sh}
\newpage
\subsection{SHELL SCRIPT MODELO DE VELOCIDADE}
\label{apendice_L}
Este shell script foi escrito para utilizar principalmente as subrotinas \textit{unisam2} e \textit{smooth2} do pacote Seismic Unix (SU).

\lstinputlisting[language=bash,breaklines=true]{codes/gera_campo_vel_bin.sh}
\subsubsection*{SCRIPT PLOT DO MODELO DE VELOCIDADE}
\lstinputlisting[language=bash,breaklines=true]{codes/plot_campo.sh}

\newpage
\subsection{SHELL SCRIPT CORREÇÃO NMO}
\label{apendice_M}
Este shell script foi escrito para utilizar principalmente a subrotina \textit{sunmo} do pacote Seismic Unix (SU).

\lstinputlisting[language=bash,breaklines=true]{codes/nmo.sh}

\newpage
\subsection{SHELL SCRIPT EMPILHAMENTO}
\label{apendice_N}
Este shell script foi escrito para utilizar principalmente a subrotina \textit{sustack} do pacote Seismic Unix (SU).

\lstinputlisting[language=bash,breaklines=true]{codes/stack.sh}
\subsubsection*{SCRIPT PLOT DA SEÇÃO EMPILHADA}
\lstinputlisting[language=bash,breaklines=true]{codes/plot_emp.sh}

\newpage
\subsection{SHELL SCRIPT MIGRAÇÃO KIRCHHOFF}
\label{apendice_O}
Este shell script foi escrito para utilizar principalmente a subrotina \textit{sumigtk} do pacote Seismic Unix (SU).

\lstinputlisting[language=bash,breaklines=true]{codes/mig_kirchhoff_time.sh}

\newpage
\subsection{SHELL SCRIPT MIGRAÇÃO DIFERENÇAS FINITAS}
\label{apendice_P}
Este shell script foi escrito para utilizar principalmente a subrotina \textit{sumigfd} do pacote Seismic Unix (SU).

\lstinputlisting[language=bash,breaklines=true]{codes/migfd.sh}


\newpage
\subsection{SHELL SCRIPT OUTRAS MIGRAÇÕES}
\label{apendice_Q}
Este shell script foi escrito para utilizar as subrotinas de migrações do pacote Seismic Unix (SU).

\lstinputlisting[language=bash,breaklines=true]{codes/migfd1.sh}
